\documentclass{beamer}


\usepackage{fontenc}

\usepackage[utf8]{inputenc}
\usepackage[english, russian]{babel}
\usepackage{graphicx}
\usepackage[inkscapeformat=png]{svg}
\usepackage{subfig}
\usepackage{minted}


\subtitle{Дипломная работа}
\title{Условия регулярности бесконтекстных L-графов}
\author{Лобанов А.М. }
\institute{Университет МГУ-ППИ в Шэньчжэне}

\usetheme{Darmstadt}

\begin{document}
	\frame {
		\titlepage
		научный руководитель: \\ к.ф-м.н. Вылиток А.А.
	}
	\frame{
		\frametitle{Регулярные языки}

		Регулярные языки описывают с помощью \emph{конечных автоматов} или \emph{регулярных выражений}.
		
		Регулярное выражение: $a^*b^+cb^*$
		
		Конечный автомат:
		\includesvg{images/nfa.dot.svg}
		
		Они описывают язык $L = \{ \alpha \in \Sigma^* \vert \alpha = a^n b^m c b^k, n,k \geq 0, m \geq 1 \}$.
	}
	\frame {
		\frametitle{L-граф}
		L-граф $G$:
		\includesvg{images/graph1.dot.svg}
		Граф $G$ описывает язык $L(G) = \{ \alpha \in \Sigma^* \vert \alpha = a^m b^{n+1} c b^n, m \geq 0, n \geq 0 \}$.
	}
	\frame {
		\frametitle{Задача}
		Определяет ли данный L-граф регулярный язык?
	}
	\frame{
		\frametitle{Пути в L-графе}
		\emph{Путь} в $T$ L-графе -- это последовательность дуг.
		\includesvg[scale=0.6]{images/graph1_path0.dot.svg}
		
		$\iota(T) = (_0 )_0$ -- скобочный след.
		
		$\omega(T) = bcb$ -- пометка пути.
	
	}
	\frame{
		\frametitle{Детерминированный L-граф}
		Недетерминированый L-граф:
		\includesvg{images/example_nondetermined.dot.svg}
		Такие графы не рассматриваем.
	}
	\frame{
		\frametitle{Циклы}
		\begin{figure}
			\centering
			\subfloat[\centering Граф с нейтральным циклом]{
				\includesvg[scale=0.5]{images/example_neutral.dot.svg}
			}
			\qquad
			\subfloat[\centering Граф с парными циклами]{
				\includesvg[scale=0.5]{images/example_paired.dot.svg}
			}
			\qquad
			\subfloat[\centering Граф с псевдоциклом]{
				\includesvg[scale=0.5]{images/example_pseudo.dot.svg}
			}
		\end{figure}
		
	}
	\frame{
		\frametitle{Ядро L-графа}
		$Core(G, w, d)$ -- это множество всех успешных путей, 
		где количество подряд идущих нейтральных циклов не превышает $w$, 
		количество вложенных подряд идущих парных циклов не превышает $d$.  

		L-граф $G$:
		\includesvg{images/graph1.dot.svg}
	}
	\frame {
		\frametitle{Ядро L-графа}
		$Core(G,1,1)$:\\
		\includesvg[scale=0.47]{images/graph1_path0.dot.svg}
		\includesvg[scale=0.47]{images/graph1_path1.dot.svg}
		\includesvg[scale=0.47]{images/graph1_path3.dot.svg}
		\includesvg[scale=0.47]{images/graph1_path4.dot.svg}
	}
	\frame {
		\frametitle{Условия регулярности}
		Если для $\forall T = T_1 T_2 T_3 T_4 T_5$, $T \in Core(G, 1, 1)$, $(T_2, T_4)$ -- простые парные циклы, 
		верно, что $\omega(T_2) = \varepsilon$ или $\omega(T_4) = \varepsilon$, то $G$ регулярен.

		L-граф $G$:

		\includesvg{images/reg.dot.svg}				

		$L(G) = \{ \alpha \in \Sigma^* \vert \alpha = a^nb, n \geq 0 \}$.

	}
	\frame {
		\frametitle{Условия регулярности}
		Если для $\forall T = T_1 T_2 T_3 T_4 T_5$, $T \in Core(G, 1, 1)$, $(T_2, T_4)$ -- простые парные циклы, 
		верно, что $\omega(T_2) = \varepsilon$ или $\omega(T_4) = \varepsilon$, то $G$ регулярен.

		$Core(G, 1, 1)$:

		\includesvg[scale=0.75]{images/reg_path1.dot.svg} \\ 
		\includesvg[scale=0.75]{images/reg_path2.dot.svg} \\ 		
	}
	\frame{
		\frametitle{Нормальная форма L-графа}
		L-граф $G$ будем считать L-графом \emph{в нормальной форме}, если в его путях нет \emph{псевдоциклов}.

		Нормальная форма полезна в докозательствах и для решения некоторых задач, таких как проверка языка на конечность.
	
	}
	\frame{
		\frametitle{Генерация нормальной формы}

		L-граф $G$:
		\includesvg{images/graph1.dot.svg}
	}
	\frame {
		$Core(G,1,1)$:\\
		\includesvg[scale=0.47]{images/graph1_path0.dot.svg}
		\includesvg[scale=0.47]{images/graph1_path1.dot.svg}
		\includesvg[scale=0.47]{images/graph1_path3.dot.svg}
		\includesvg[scale=0.47]{images/graph1_path4.dot.svg}
	}
	\frame{
		\frametitle{Генерация нормальной формы}
		L-граф $G_0$:
		\includesvg[scale=0.4]{images/graph1_normal_g0.dot.svg}
	}
	\frame{
		\frametitle{Генерация нормальной формы}
		$Core(G,1,2) \setminus Core(G,1,1)$: \\
		\includesvg[scale=0.35]{images/graph1_path2.dot.svg}
		\includesvg[scale=0.35]{images/graph1_path5.dot.svg}
	}
	\frame{
		\frametitle{Генерация нормальной формы}
		L-граф $G' \sim  G$:
		\includesvg[scale=0.4]{images/graph1_normal.dot.svg}
	}
	\frame {
		\frametitle{Програмная библиотека}
		\inputminted[linenos,fontsize=\tiny]{rust}{../lgraphs/examples/presentation.rs}
	}
\end{document}