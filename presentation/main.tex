\documentclass[12pt]{beamer}

\usepackage{fontenc}
\usepackage[utf8]{inputenc}
\usepackage[english, russian]{babel}
\usepackage{graphicx}
\usepackage[inkscapeformat=png]{svg}
\usepackage{subfig}
\usepackage{minted}

\usetheme[progressbar=frametitle]{metropolis}
\setbeamercolor{background canvas}{bg=white}

\setbeamertemplate{title page}{
    \begin{minipage}[c][\paperheight]{\textwidth}
        \ifx\inserttitlegraphic\@empty\else\usebeamertemplate*{title graphic}\fi
        \vfill%
        {
        \centering
        \ifx\inserttitle\@empty\else\usebeamertemplate*{title}\fi
        \ifx\insertsubtitle\@empty\else\usebeamertemplate*{subtitle}\fi
        }
        \usebeamertemplate*{title separator}
        \begin{minipage}[t]{.5\textwidth}
            \ifx\beamer@shortauthor\@empty\else\usebeamertemplate*{author}\fi
            \ifx\insertdate\@empty\else\usebeamertemplate*{date}\fi
        \end{minipage}
        \begin{minipage}[t]{.5\textwidth}
            \vspace*{2em}
            {\hspace{3.2em}\small Научный руководитель: \par }
			{\hspace{3.2em}Вылиток А.А. }
        \end{minipage}%

        \begin{minipage}[t]{\textwidth}
            \ifx\insertinstitute\@empty\else\usebeamertemplate*{institute}\fi
        \end{minipage}
        \vfill
        \vspace*{1mm}
    \end{minipage}
}


\setbeamertemplate{title}{
%  \raggedright%  % <-- Comment here
  \linespread{1.0}%
  \inserttitle%
  \par%
  \vspace*{0.5em}
}
\setbeamertemplate{subtitle}{
%  \raggedright%  % <-- Comment here
  \insertsubtitle%
  \par%
  \vspace*{0.5em}
}


\title{Условия регулярности бесконтекстных L-графов}
\subtitle{Дипломная работа}
\author{Лобанов А.М. }
\institute{Университет МГУ-ППИ в Шэньчжэне}


\begin{document}
	\maketitle
	\frame{
		\frametitle{Регулярные языки}

		Регулярные языки описывают с помощью \emph{конечных автоматов} или \emph{регулярных выражений}.
		
		Регулярное выражение: $a^*b^+cb^*$
		
		Конечный автомат:
		\includesvg{images/nfa.dot.svg}
		
		Они описывают язык $L = \{ a^n b^m c b^k \vert n,k \geq 0, m \geq 1 \}$.
	}
	\frame {
		\frametitle{L-граф}
		L-граф $G$:
		\includesvg{images/graph1.dot.svg}
		Граф $G$ описывает язык $L(G) = \{ a^m b^{n+1} c b^n \vert m \geq 0, n \geq 0 \}$.
	}
	\frame {
		\frametitle{Отличие от конечного автомата}

		\includesvg[scale=0.6]{images/graph1_path0.dot.svg} \\
		\includesvg[scale=0.6]{images/graph1_path_bad.dot.svg} \\
		$\iota(T_1) = (_0 )_0$ -- скобочный след. Скобки сбалансированы, путь успешен. \\
		$\iota(T_2) = (_0$ -- несбалансированные скобки, путь не успешен. \\
		$\omega(T_1) = bcb$ -- пометка пути. \\

		\includesvg[scale=0.6]{images/graph1.dot.svg} $L(G) = \{ \omega(T) \vert T \text{-- успешный путь в }G \}$
	}
	\frame {
		\frametitle{Задача}
		Определяет ли данный L-граф регулярный язык?
		
		Найти достаточные условия регулярности бесконтекстного L-графа.
	}
	\frame {
		\frametitle{Память}
		\emph{Память} описывает состояние L-графа, и является парой $\langle q, \alpha \rangle$, $q$ -- вершина, $\alpha$ -- стек скобок.

		Путь в форме памяти:
		\includesvg[scale=0.6]{images/graph1_path0_mem.dot.svg}

	}
	\frame{
		\frametitle{Циклы}
		\begin{figure}
			\centering
			\subfloat[\centering Граф с нейтральным циклом]{
				\includesvg[scale=0.5]{images/example_neutral.dot.svg}
			}
			\qquad
			\subfloat[\centering Граф с парными циклами]{
				\includesvg[scale=0.5]{images/example_paired.dot.svg}
			}
			\qquad
			\subfloat[\centering Граф с псевдоциклом]{
				\includesvg[scale=0.5]{images/example_pseudo.dot.svg}
			}
		\end{figure}
		
	}
	\frame{
		\frametitle{Ядро L-графа}
		$Core(G, w, d)$ -- это множество всех успешных путей, 
		где количество подряд идущих нейтральных циклов не превышает $w$, 
		количество вложенных подряд идущих парных циклов не превышает $d$.  

		L-граф $G$:
		\includesvg{images/graph1.dot.svg}
	}
	\frame {
		\frametitle{Ядро L-графа}
		L-граф $G$:\\
		\includesvg[scale=0.47]{images/graph1.dot.svg}\\
		$Core(G,1,1)$:\\
		\includesvg[scale=0.47]{images/graph1_path0.dot.svg}
		\includesvg[scale=0.47]{images/graph1_path1.dot.svg}
		\includesvg[scale=0.47]{images/graph1_path3.dot.svg}
		\includesvg[scale=0.47]{images/graph1_path4.dot.svg}

	}
	\frame {
		\frametitle{Условия регулярности}
		В L-графе $G$ нет дуг с буквами $\implies$ $G$ регулярный. 
		\includesvg[scale=0.5]{images/reg_cond_1.dot} $L = \{ \varepsilon \}$

		В L-графе $G$ нет дуг со скобками $\implies$ $G$ регулярный.
		\includesvg[scale=0.5]{images/reg_cond_2.dot} $a^*b^*$

		Пусть $Core(G, 1, 0) = Core(G, 1, 1)$. тогда:
		\begin{enumerate}
			\item $\forall d = 0,1,2 \dots Core(G, 1, d) = Core(G, 1, 0)$.
			\item $L(G)$ -- регулярный.
		\end{enumerate}
		\includesvg[scale=0.5]{images/reg_cond_3.dot.svg} $ab^*$
	}
	\frame {
		\frametitle{Условия регулярности}
		Если для $\forall T = T_1 T_2 T_3 T_4 T_5$, $T \in Core(G, 1, 1)$, $(T_2, T_4)$ -- простые парные циклы, 
		верно, что $\omega(T_2) = \varepsilon$ или $\omega(T_4) = \varepsilon$, то $G$ регулярен.

		L-граф $G$:

		\includesvg{images/reg_cond_4.dot.svg}				

		$L(G) = \{ a^nb \vert n \geq 0 \}$.

	}
	\frame {
		\frametitle{Условия регулярности}
		L-граф $G$:\\
		\includesvg[scale=0.75]{images/reg_cond_4.dot.svg}				

		$Core(G, 1, 1)$:

		\includesvg[scale=0.75]{images/reg_path1.dot.svg} \\ 
		\includesvg[scale=0.75]{images/reg_path2.dot.svg} \\ 		
		$L(G) = \{ a^nb \vert n \geq 0 \}$ соответствует регулярному выражению $a^*b$.
	}
	\frame{
		\frametitle{Нормальная форма L-графа}
		L-граф $G$ будем считать L-графом \emph{в нормальной форме}, если в его путях нет \emph{псевдоциклов}.

		Нормальная форма полезна в доказательствах и для решения некоторых задач, таких как проверка языка на конечность или на пустоту.
	
	}
	\frame{
		\frametitle{Приведение к нормальной форме}

		L-граф $G$:
		\includesvg{images/graph1.dot.svg}
	}
	\frame {
		\frametitle{Приведение к нормальной форме}
		$Core(G,1,1)$:\\
		\includesvg[scale=0.47]{images/graph1_path0.dot.svg}
		\includesvg[scale=0.47]{images/graph1_path1.dot.svg}
		\includesvg[scale=0.47]{images/graph1_path3.dot.svg}
		\includesvg[scale=0.47]{images/graph1_path4.dot.svg}
		L-граф $G_0$:
		\includesvg[scale=0.4]{images/graph1_normal_g0.dot.svg}
	}
	\frame{
		\frametitle{Приведение к нормальной форме}
		$Core(G,1,2) \setminus Core(G,1,1)$: \\
		\includesvg[scale=0.35]{images/graph1_path2.dot.svg} \\
		\includesvg[scale=0.35]{images/graph1_path5.dot.svg} \\
		L-граф $G'$: \\
		\includesvg[scale=0.4]{images/graph1_normal.dot.svg}
	}
	\frame{
		\frametitle{Приведение к нормальной форме}
		L-граф $G$:\\
		\includesvg[scale=0.4]{images/graph1.dot.svg}\\
		L-граф $G' \sim  G$:\\
		\includesvg[scale=0.4]{images/graph1_normal.dot.svg}
	}
	\frame {
		\frametitle{Програмная библиотека}
		\inputminted[linenos,fontsize=\tiny]{rust}{../lgraphs/examples/presentation.rs}
	}
	\begin{frame}[fragile]
		\frametitle{Програмная библиотека}
		Вывод программы для графа $G$:\\
		\includesvg[scale=0.7]{images/reg_cond_4.dot.svg}\\
		\begin{small}
			\begin{verbatim}
				$ ->1;2->;1-a,[1->1;1-b->2;2-]1->2;
				$ 1-b->2
				$ 1-a,[1->1-b->2-]1->2
				$ ->1,{};2,{}->;1,{}-a,[1->1,{1};1,{}-b->2,{};2,{1}-]1->2,{};
						1,{1}-b->2,{1};1,{1}-a,[1->1,{1};2,{1}-]1->2,{1};
				$ Does the graph not have letters on paired loops? true
			\end{verbatim}
		\end{small}

	\end{frame} 
	\frame{
		\frametitle{Результаты}
		\begin{enumerate}
			\item Найдены алгоритмически реализуемые достаточные условия регулярности L-графов.
			\item Разработан алгоритм построения ядер L-графа.
			\item Разработан алгоритм приведения L-графа к нормальной форме.
			\item Создана программная библиотека, реализующая все разработанные алгоритмы и проверку условий регулярности.
		\end{enumerate}
	}
\end{document}