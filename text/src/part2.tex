\chapter{Описание алгоритмов} \label{chapter2} 

\section{Построение ядра}
Для построения ядра нам понадобются некоторые факты, доказанные в \cite{stan1}:

\begin{theorem}
    \label{canon_length_limit_theorem}
    Пусть $w$, $d$ -- неотриательные целые. Пусть последовательность скобок $P : \mu(P)=\varepsilon$, $depth(P) \leq d$, $width(P) \leq w$. 
    Тогда длина $P$ ограниченна сверху числом $g_{w,d}$, где 
    \begin{enumerate}
        \item $g_{w,1} = 2w$;
        \item $g_{w,d} = (g_{w,d-1} + 2)w$ .
    \end{enumerate}
\end{theorem}
\begin{lemma}
    \label{wd_upper_limit_lemma}
    Пусть $m$ -- число вершин L-графа. Тогда, глубина скобочного следа $(w,d)$-канона ограничена сверху числом $(d+1)m^2$, ширина ограничена $(d+1)m$.
\end{lemma}

\subsection*{Алгоритм генерации ядра}

\begin{algorithmic}
    \Function{CoreRec}{$G$, $w_0$, $maxDepth$, $T$}
        \State {$q \gets end(T)$}
        \State {$paths \gets \emptyset$}
        \If {$end(T) \in Final(G) \wedge \mu(\iota(T)) = \varepsilon$}
            \State {$paths \gets paths \cup T$}
        \EndIf

        \ForAll{$\pi \in Edges(G) : beg(\pi) = q$}
            \State {$T_1 \gets T \pi$}

            \If {$\iota(T_1)$ -- не префикс пути в $L_p$}
                continue
            \ElsIf {$depth(\iota(T_1)) > maxDepth$}
                continue
            \ElsIf {$\text{параметр } w \text{ маршрута } T_1 > w_0$} 
                continue
            \EndIf
            
            \State {$paths \gets paths \cup \Call{CoreRec}{G, w_0, maxDepth, T_1}$}
        \EndFor

        \State \Return $paths$
    \EndFunction
    \\
    \Function{Core}{$G$, $w$, $d$}
        \State {$m \gets$ количество вершин в $G$ }
        \State {$q_0 \gets Init(G)$}
        \Comment {В детременированном L-графе начальная вершина только одна}
        \State {$D \gets \emptyset$}
        \ForAll{$T \in \Call{CoreRec}{G, w, (d+1)m^2, \text{пустой путь из } q_0}$}
            \If{$T$ -- $(w,d)$-канон}
                \State {$D \gets D \cup path$}
            \EndIf
        \EndFor
        \State \Return $D$
    \EndFunction
\end{algorithmic}

\begin{theorem}
    Алгоритм генерации ядра корректен.
\end{theorem}

\begin{proof}
    Требуется сгенерировать $Core(w_0, d_0)$ L-графа $G$. 
    Пусть нам дано множество $D_0$, содеражащее все успешные пути, 
    с глубиной скобочного следа, ограниченной сверху $(d_0+1)m^2 = maxDepth$, 
    и параметром $w$, ограниченным сверху $w_0$. 
    $D_0$ -- конечное множество, по теореме \ref{canon_length_limit_theorem}.
    Тогда, по лемме \ref{wd_upper_limit_lemma} можно просто отобрать $(w,d)$-каноны из этого множества перебором, 
    и получить ядро. 
    Это и делает функция $Core(G, w, d)$, поэтому требуется доказать только то, 
    что множество $D = CoreRec(G, w_0, (d+1)m^2, q_0)$ равно множеству $D_0$, и что $D$ получается за конечное время.

    Пусть путь $T \in D$. В цикле есть проверка $depth(\iota(T)) \leq maxDepth$, проверка параметра $w$, и проверка на успешность пути.
    Это значит, что $T \in D_0$.

    Теперь, пусть $T_0 \in D_0$, но $T_0 \notin D$, то есть либо $T_0$, либо его префикс $T_1$, был отброшен в ходе проверок.
    Если сам $T_0$ отброшен, то это очевидно значит, что он не принадлежит $D_0$.
    Если отброшен был $T_1$, то это могло произойти по трем причинам:
    \begin{enumerate}
        \item В $T_1$ слишком много закрывающих скобок, но это значит, что и $T_0$ не может быть успешным;
        \item $depth(\iota(T_1)) \ge maxDepth$, но из определения функции $depth$ следует, что $depth(T_0) \geq depth(T_1) \ge maxDepth$;
        \item Параметр $w$ $T_1$ превышает $w_0$. Это противоречит определению $(w,d)$-канона.
    \end{enumerate}
    Это значит, что $T_0 \in D$. $D = D_0$.

    Для доказательства завершаемости алгоритма можно заметить следующее:
    \begin{enumerate}
        \item Множество дуг в L-графе конечно;
        \item Маршруты, которые не подходят по $w$ или глубине отбрасываются сразу, и маршруты, чьими префиксами они являются, не рассматриваются;
        \item При каждом вызове $CoreRec$ получает разные аргументы $path$, это значит, что каждый путь генерируется только один раз.
    \end{enumerate}

    Итого, функция $CoreRec$ верна, завершается за конечное время, а значит и весь алгоритм корректен. \qed
\end{proof}

\section{Нормальная форма}

\begin{definition}
    Нормальная форма L-графа $G$ -- L-граф $G'$, эквивалентный $G$, в путях которого отсутсвуют псевдоциклы.
\end{definition}

\subsection*{Алгоритм генерации нормальной формы}
\begin{algorithmic}
    \Function{MemGraph}{paths}
        \State {$E \gets \bigcup_{T \in paths} \bigcup_{\pi \in Edges(Mem(T))} \pi $}
        \State Очевидно, что по информации, хранящейся в дугах $E$, можно однозначно построить граф $G$.
        \State \Return G
    \EndFunction
    
    \\
    \Require{$T_l, T_r$ в форме памяти}
    \Function{AddLoops}{$G', T_l, T_r$}
        \State Добавим каждую дугу из $T_l$ в $G'$, но конечную вершину переименуем в начальную.
        \State То же самое для $T_r$, но теперь переиминуем начальную в конечную, а конечную не переименовываем.

        \State \Return $G'$
    \EndFunction
    \\
    \Require{$d_0 \geq 1$}
    \Function{NormalForm}{$G$, $d_0$}
        \State {$G_0 \gets MemGraph(Core(1, d_0))$}
        \State {$G' \gets G_0$}

        \ForAll{$T \in Core(G, 1, d_0+1) \setminus Core(G, 1, d_0)$}
            \ForAll {$T_1 T_2 T_3 T_4 T_5 = T$, $\langle T_2, T_3, T_4\rangle$ -- простое гнездо, $T_1 T_3 T_5 \in Core(G, 1, d_0)$}
                \State $T_l \gets Mem(T_1 T_2)$
                \State $T_r \gets Mem(T_1 T_2 T_3 T_4)$
                \If{$beg(T_l) \in Nodes(G_0) \wedge end(T_r) \in Nodes(G_0)$} 
                    \State{$G' \gets \Call{AddLoops}{G', T_l, T_r}$}
                \EndIf
            \EndFor
        \EndFor

        \State \Return {$G'$}
    \EndFunction
\end{algorithmic}

\begin{theorem}
    В путях $G' = NormalForm(G, d_0)$ нет псевдоциклов.
\end{theorem}

\begin{proof}
    Достаточно рассмотреть только пути из $Core(G',w,1)$, так как если путь из $Core(G',w,0)$, в нем очевидно 
    по построению не может быть псевдоциклов, а из путей $Core(G',w,d), d > 1$, всегда можно убрать
    некоторые парные циклы, не повлияв на наличие псевдоциклов.

    Пусть $T = T_1 T_2 T_3 T_4 T_5 \in Core(G',w,1)$, $\langle T_2, T_3, T_4 \rangle$ -- гнездо с псевдоциклом,
    в $T_1$ и $T_5$ нет других циклов,
    для определенности, пусть $T_2$ -- цикл.
    По построению, $T_2$ и $T_4$ не могли быть добавлены в $G'$ одним путем, значит 
    $\exists T' = T_a T_b T_3 T_4 T_5$, $\langle T_b, T_3, T_4 \rangle$ -- гнездо, $T_b$ -- не цикл,
    $\mu(\iota(T_a)) = \alpha$,
    $\mu(\iota(T_b)) = \mu(\iota(T_2)) = \beta$,
    в $T_a T_b$ нет других циклов.
    Это значит, что $beg(T_3) = \langle q, \alpha \beta \rangle$, но $end(T_1) = end(T_2) = beg(T_3)$,
    значит $\mu(\iota(T_1)) = \alpha \beta$, и $\exists T'' = T_1 T_3 T_4 T_5$.
    Таким образом, $T_2$ -- нейтральный маршрут, что пртиворечит определению псевдоцикла, 
    значит, в путях $G'$ не может быть псевдоциклов. $\qed$ 
\end{proof}

\begin{theorem}
    $G' = NormalForm(G, d_0)$ эквивалентен $G$.
\end{theorem}

\begin{proof}
    Докажем, что язык $L' = L(G')$ эквивалентен $L = L(G)$.
    
    Сначала докажем, что $L \subseteq L'$.
    $\forall T : \omega(T) \in L \implies \exists w, d : T \in Core(G, w, d)$.
    Рассмотрим три случая, для путей принадлежащих ядрам $Core(1, d), d \geq 0$:

    Для $T \in Core(1, d_0)$ очевидно, что $\omega(T) \in L'$, так как все пути $Core(1, d_0)$ были явно добавлены в граф $G_0$.

    Пусть $T = T_1 T_2 T_3 T_4 T_5 \in Core(1, d_0 + 1) \setminus Core(1, d_0)$, 
    $\langle T_2, T_3, T_4 \rangle$ -- простое гнездо, $(T_2, T_4)$ -- парные циклы,
    $T_1 = T_{1, 0} T_{1, 1} \cdots T_{1, d_0}$,
    $T_5 = T_{5, d_0} \cdots T_{5, 1}, T_{5, 0}$,
    $(T_{1, i}, T_{5, i}), i = 1 \dots d_0$ -- парные.
    Будем считать, что в $T_{1,0}$ и $T_{5,0}$ нет других парных циклов,
    потому что их можно отбросить, не повлияв на то, что $T \in Core(1, d_0+1)$. 
    Нужно доказать, что функция $AddLoops$ присоеденила цикл $T_2$ к $end(Mem(T_1))$, 
    и $T_4$ к $end(Mem(T_1 T_2 T_3 T_4))$.
    Очевидно, что $\exists T_0 = T_1 T_3 T_5 \in Core(1, d_0)$, то есть $Mem(T_0)$ -- путь в $G_0$. 
    $end(Mem(T_1)) \in Nodes(G_0)$, $end(Mem(T_1 T_2 T_3 T_4)) = end(Mem(T_1 T_3)) \in Nodes(G_0)$.

    Более общий случай: $T = T_1 T_2 T_3 T_4 T_5 T_6 T_7\in Core(1, d), d > d_0 + 1$.
    Будем считать, что пути $T' = T_1 T_2 T_4 T_6 T_7$ и $T'' = T_1 T_3 T_4 T_5 T_7 \in Core(1, d_0+1)$
    (если это не так, $T$ всегда можно аналогичным образом разбить на еще более простые пути $T^i$,
    каждый из которых будет в $Core(1, d_0+1)$).
    Тогда очевидно, что парные циклы $(T_2,T_6), (T_3,T_5)$ были добавлены в $G'$, и значит $T$ -- путь в $G'$. 

    По работе функции $MemGraph$ очевидно, что при $w \neq 1$ все пути так же будут успешными.
    Это значит, что $L \subseteq L'$.

    Теперь докажем, что $L' \subseteq L$. 
    Очевидно, что пути из $Core(G', w, 0)$ однозначно соответсвуют путям из $Core(G, w, d_0)$. 
    Пусть $T = T_1 T_2 T_3 T_4 T_5 \in Core(G', w, 1)$, 
    $\langle T_2 T_3 T_4 \rangle$ -- простое гнездо, $(T_2, T_4)$ -- парные циклы.
    Будем считать, что в $T_1$ и $T_5$ нет других парных циклов,
    потому что их можно отбросить, не повлияв на то, что $T \in Core(G',w,1)$. 
    Попобуем сконструировать путь $T' = T_1' T_2' T_3' T_4' T_5', Mem(T') = T$, 
    $Mem(T_1') = T_1, Mem(T_1', T_2') = T_2$ и т.д., $\omega(T') \in L$.
    Во первых, в $Core(G, w, d_0 + 1)$ есть путь $T'' = T_1' T_2' T_x T_4' T_y$, $\mu(\iota(T_x)) = \varepsilon$, 
    который добавил в $G'$ парные циклы $T_2$ и $T_4$.
    Это значит, что память $end(T_1)$ можно продолжить маршрутом $T_2'$, 
    а $\langle end(T_x), \mu(\iota(T_1' T_2')) \rangle$ можно продолжить маршрутом $T_4'$.
    Во вторых, в $Core(G, w, d_0)$ очевидно существует путь $T''' = T_1' T_3' T_5'$, $\mu(\iota(T_3')) = \varepsilon$, 
    то есть память $\langle end(T_3'), \mu(\iota(T_1')) \rangle$ можно продолжить маршрутом  $T_5'$. 
    $T_3'$ -- нейтральный маршрут, то есть им можно продолжить любую память вершины $beg(T_3')$, 
    $end(T_1') = beg(T_3')$, $end(T_3') = beg(T_4') = end(T_x) = end(T_4')$, а значит 
    $\exists T' = T_1' T_2' T_3' T_4' T_5'$ -- путь в $Core(G, w, d_0 + 1)$, 
    соответсвующий $T$. Аналогично можно доказать, что $T \in Core(G', w, d) \implies \omega(T) \in L, \forall d \geq 1$.

    Таким образом, $L' = L$, то есть $G'$ эквивалентен $G$.
    \qed 
    

\end{proof}

\clearpage