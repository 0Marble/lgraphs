\chapter{Алгоритмы обработки L-графов} \label{chapter2} 

% cSpell:ignore завершаемости детреминированном псевдоцикла псевдоциклы псевдоциклом

\section{Построение ядра}
\label{core_gen_algo}

Для построения ядра нам понадобятся некоторые факты, доказанные в \cite{stan1}:

\begin{theorem}
    \label{canon_length_limit_theorem}
    Пусть $w$, $d$ -- неотрицательные целые. Пусть последовательность скобок $P : \mu(P)=\varepsilon$, $depth(P) \leq d$, $width(P) \leq w$. 
    Тогда длина $P$ ограниченна сверху числом $g_{w,d}$, где 
    \begin{enumerate}
        \item $g_{w,1} = 2w$;
        \item $g_{w,d} = (g_{w,d-1} + 2)w$ .
    \end{enumerate}
\end{theorem}
\begin{lemma}
    \label{wd_upper_limit_lemma}
    Пусть $m$ -- число вершин L-графа. Тогда, глубина скобочного следа $(w,d)$-канона ограничена сверху числом $(d+1)m^2$, ширина ограничена $(d+1)m$.
\end{lemma}

\subsection*{Алгоритм генерации ядра}

\begin{algorithmic}[1]
    \Function{CoreRec}{$G$, $w_0$, $maxDepth$, $T$}
        \State {$q \gets end(T)$}
        \State {$paths \gets \emptyset$}
        \If {$end(T) \in Final(G) \wedge \mu(\iota(T)) = \varepsilon$}
            \State {$paths \gets paths \cup T$}
        \EndIf

        \ForAll{$\pi \in Edges(G) : beg(\pi) = q$}
            \State {$T_1 \gets T \pi$}

            \If {$\iota(T_1)$ -- не префикс пути в $L_p$}
                continue \Comment{переход на следующую итерацию цикла}
            \ElsIf {$depth(\iota(T_1)) > maxDepth$}
                continue
            \ElsIf {$\text{параметр } w \text{ маршрута } T_1 > w_0$} 
                continue
            \EndIf
            
            \State {$paths \gets paths \cup \Call{CoreRec}{G, w_0, maxDepth, T_1}$}
        \EndFor

        \State \Return $paths$
    \EndFunction
    \\
    \Function{Core}{$G$, $w$, $d$}
        \State {$m \gets$ количество вершин в $G$ }
        \State {$q_0 \gets Init(G)$}
        \Comment {В детреминированном L-графе начальная вершина только одна}
        \State {$D \gets \emptyset$}
        \ForAll{$T \in \Call{CoreRec}{G, w, (d+1)m^2, \text{пустой путь из } q_0}$}
            \If{$T$ -- $(w,d)$-канон}
                \State {$D \gets D \cup path$}
            \EndIf
        \EndFor
        \State \Return $D$
    \EndFunction
\end{algorithmic}

\begin{theorem}
    Алгоритм генерации ядра корректен.
\end{theorem}

\begin{proof}
    Требуется сгенерировать $Core(w_0, d_0)$ L-графа $G$. 
    Пусть нам дано множество $D_0$, содержащее все успешные пути, 
    с глубиной скобочного следа, ограниченной сверху $(d_0+1)m^2 = maxDepth$, 
    и параметром $w$, ограниченным сверху $w_0$. 
    $D_0$ -- конечное множество, по теореме \ref{canon_length_limit_theorem}.
    Тогда, по лемме \ref{wd_upper_limit_lemma} можно просто отобрать $(w,d)$-каноны из этого множества перебором, 
    и получить ядро. 
    Это и делает функция $Core(G, w, d)$, поэтому требуется доказать только то, 
    что множество $D = CoreRec(G, w_0, (d+1)m^2, q_0)$ равно множеству $D_0$, и что $D$ получается за конечное время.

    Пусть $T_0 \in D_0$, но $T_0 \notin D$, то есть либо $T_0$, либо его префикс $T_1$, был отброшен в ходе проверок.
    Если сам $T_0$ отброшен, то это очевидно значит, что он не принадлежит $D_0$.
    Если отброшен был $T_1$, то это могло произойти по трём причинам.
    \begin{enumerate}
        \item В $T_1$ слишком много закрывающих скобок, но это значит, что и $T_0$ не может быть успешным.
        \item $depth(\iota(T_1)) \ge maxDepth$, но $depth(T_0) \geq depth(T_1) \ge maxDepth$.
        \item Параметр $w$ $T_1$ превышает $w_0$. Это противоречит определению $(w,d)$-канона.
    \end{enumerate}
    Это значит, что $T_0 \in D$. $D_0 \subseteq D$.

    Теперь, пусть $T \in D$. Любой префикс $T$ проходит по условиям в строках 9-12. Так как $beg(T) = q_0$ -- начальной вершине $G$,
    то очевидно, что $T \in D_0$. $D \subseteq D_0$

    Для доказательства завершаемости алгоритма можно заметить следующее.
    \begin{enumerate}
        \item Множество дуг в L-графе конечно.
        \item Маршруты, которые не подходят по $w$ или глубине отбрасываются сразу, и маршруты, чьими префиксами они являются, не рассматриваются.
        \item При каждом вызове $CoreRec$ получает разные аргументы $path$, это значит, что каждый путь генерируется только один раз.
    \end{enumerate}

    Итого, функция $CoreRec$ верна, завершается за конечное время, а значит и весь алгоритм корректен $\Box$ 
\end{proof}

\section{Нормальная форма}
\label{normal_gen_algo}

\subsection*{Алгоритм генерации нормальной формы}
\begin{algorithmic}[1]
    \Function{MemGraph}{paths}
        \State {$E \gets \bigcup_{T \in paths} \bigcup_{\pi \in Edges(Mem(T))} \{\pi\} $}
        \State Очевидно, что по информации, хранящейся в дугах $E$, можно однозначно построить граф $G$.
        \State \Return G
    \EndFunction
    
    \\
    \Function{AddLoops}{$G', T_l, T_r$}
        \State Добавим каждую дугу из $T_l$ в $G'$, но конечную вершину переименуем в начальную.
        \State То же самое для $T_r$, но теперь переименуем начальную в конечную, а конечную не переименовываем.

        \State \Return $G'$
    \EndFunction
    \\
    \Function{NormalForm}{$G$}
        \State {$G_0 \gets MemGraph(Core(1, 1))$}
        \State {$G' \gets G_0$}

        \ForAll{$T \in Core(G, 1, 2) \setminus Core(G, 1, 1)$}
            \ForAll {$T_1 T_2 T_3 T_4 T_5 = T$, $\langle T_2, T_3, T_4\rangle$ -- простое гнездо, $T_1 T_3 T_5 \in Core(G, 1, 1)$}
                \State $T_l \gets Mem(T_1, T_2)$
                \State $T_r \gets Mem(T_1 T_2 T_3, T_4)$
                \If{$beg(T_l) \in Nodes(G_0) \wedge end(T_r) \in Nodes(G_0)$} 
                    \State{$G' \gets \Call{AddLoops}{G', T_l, T_r}$}
                \EndIf
            \EndFor
        \EndFor

        \State \Return {$G'$}
    \EndFunction
\end{algorithmic}

\begin{theorem}
    В путях $G' = NormalForm(G)$ нет псевдоциклов.
\end{theorem}

\begin{proof}
    Достаточно рассмотреть только пути из $Core(G',w,1)$, так как если путь из $Core(G',w,0)$, в нем очевидно 
    по построению не может быть псевдоциклов, а из путей $Core(G',w,d)$, $d > 1$, всегда можно убрать
    некоторые парные циклы, не повлияв на наличие псевдоциклов.

    Пусть $T = T_1 T_2 T_3 T_4 T_5 \in Core(G',w,1)$, $\langle T_2, T_3, T_4 \rangle$ -- гнездо с псевдоциклом,
    в $T_1$ и $T_5$ нет других циклов,
    для определенности, пусть $T_2$ -- цикл.
    По построению, $T_2$ и $T_4$ не могли быть добавлены в $G'$ одним путем, значит 
    $\exists T' = T_a T_b T_3 T_4 T_5$, $\langle T_b, T_3, T_4 \rangle$ -- гнездо, $T_b$ -- не цикл,
    $\mu(\iota(T_a)) = \alpha$,
    $\mu(\iota(T_b)) = \mu(\iota(T_2)) = \beta$,
    в $T_a T_b$ нет других циклов.
    Это значит, что $beg(T_3) = \langle q, \alpha \beta \rangle$, но $end(T_1) = end(T_2) = beg(T_3)$,
    значит $\mu(\iota(T_1)) = \alpha \beta$, и $\exists T'' = T_1 T_3 T_4 T_5$.
    Таким образом, $T_2$ -- нейтральный маршрут, что противоречит определению псевдоцикла, 
    значит, в путях $G'$ не может быть псевдоциклов $\Box$ 
\end{proof}

\begin{theorem}
    $G' = NormalForm(G)$ эквивалентен $G$.
\end{theorem}

\begin{proof}
    Докажем, что язык $L' = L(G')$ эквивалентен $L = L(G)$.
    
    Сначала докажем, что $L \subseteq L'$.
    $\forall T : \omega(T) \in L \implies \exists w, d : T \in Core(G, w, d)$.
    Если удастся доказать, что $\forall T \in Core(G, 1, 2) \implies \omega(T) \in L'$, то 
    это будет верно и для $\forall T \in Core(G, w, d), w > 1, d > 2$.

    Для $T \in Core(G, 1, 1)$ очевидно, что $\omega(T) \in L'$, так как все пути $Core(G, 1, 1)$ были явно добавлены в граф $G_0$.

    Пусть $T = T_1 T_2 T_3 T_4 T_5 \in Core(G, 1, 2) \setminus Core(G, 1, 1)$, 
    $\langle T_2, T_3, T_4 \rangle$ -- простое гнездо, $(T_2, T_4)$ -- парные циклы,
    $T_1 = T_{1, 0} T_{1, 1}$,
    $T_5 = T_{5, 1} T_{5, 0}$,
    $(T_{1, 1}, T_{5, 1})$ -- простые парные циклы.
    Будем считать, что в $T_{1,0}$ и $T_{5,0}$ нет других парных циклов,
    потому что их можно отбросить, не повлияв на то, что $T \in Core(1, 2)$. 
    Нужно доказать, что функция $AddLoops$ присоединила цикл $T_2$ к $end(Mem(T_1))$, 
    и $T_4$ к $end(Mem(T_1 T_2 T_3 T_4))$.
    Очевидно, что $\exists T_0 = T_1 T_3 T_5 \in Core(G, 1, 1)$, то есть $Mem(T_0)$ -- путь в $G_0$. 
    $end(Mem(T_1)) \in Nodes(G_0)$, $end(Mem(T_1 T_2 T_3 T_4)) = end(Mem(T_1 T_3)) \in Nodes(G_0)$.

    Это значит, что $L \subseteq L'$.

    Теперь докажем, что $L' \subseteq L$. 
    Очевидно, что пути из $Core(G', w, 0)$ однозначно соответствуют путям из $Core(G, w, 1)$. 
    Пусть $T = T_1 T_2 T_3 T_4 T_5 \in Core(G', w, 1)$, 
    $\langle T_2 T_3 T_4 \rangle$ -- простое гнездо, $(T_2, T_4)$ -- парные циклы.
    Будем считать, что в $T_1$ и $T_5$ нет других парных циклов,
    потому что их можно отбросить, не повлияв на то, что $T \in Core(G',w,1)$. 
    Попробуем сконструировать путь $T' = T_1' T_2' T_3' T_4' T_5', Mem(T') = T$, 
    $Mem(T_1') = T_1, Mem(T_1', T_2') = T_2$ и т.д., $\omega(T') \in L$.
    Во первых, в $Core(G, w, 2)$ есть путь $T'' = T_1' T_2' T_x T_4' T_y$, $\mu(\iota(T_x)) = \varepsilon$, 
    который добавил в $G'$ парные циклы $T_2$ и $T_4$.
    По \ref{mem_continue_lemma} это значит, что память $end(T_1)$ можно продолжить маршрутом $T_2'$, 
    а $\langle end(T_x), \mu(\iota(T_1' T_2')) \rangle$ можно продолжить маршрутом $T_4'$.
    Во вторых, в $Core(G, w, 1)$ очевидно существует путь $T''' = T_1' T_3' T_5'$, $\mu(\iota(T_3')) = \varepsilon$, 
    то есть память $\langle end(T_3'), \mu(\iota(T_1')) \rangle$ можно продолжить маршрутом  $T_5'$. 
    $T_3'$ -- нейтральный маршрут, то есть им можно продолжить любую память вершины $beg(T_3')$, 
    $end(T_1') = beg(T_3')$, $end(T_3') = beg(T_4') = end(T_x) = end(T_4')$, а значит 
    $\exists T' = T_1' T_2' T_3' T_4' T_5'$ -- путь в $Core(G, w, 2)$, 
    соответствующий $T$. Аналогично можно доказать, что $T \in Core(G', w, d) \implies \omega(T) \in L, \forall d \geq 1$.

    Таким образом, $L' = L$, то есть $G'$ эквивалентен $G$ $\Box$ 
    

\end{proof}

\section{Условия регулярности}

Давайте разберем несколько условий регулярности, которые можно придумать для L-графов.
Вначале рассмотрим самые очевидные условия.

\begin{theorem}
    \label{condition_no_letters}
    В L-графе $G$ нет дуг с буквами $\implies$ $G$ регулярный. 
\end{theorem}
Тогда $L(G) = \{ \varepsilon \}$ -- одна пустая строка.


\begin{theorem}
    \label{condition_no_brackets}
    В L-графе $G$ нет дуг со скобками $\implies$ $G$ регулярный.
\end{theorem}
Тогда $G$ является конечным автоматом.

Дальше нам понадобится лемма - критерий регулярности, доказательство которой можно посмотреть в \cite{handbook_of_formal}.

\begin{lemma}
    \label{pumping_lemma_4_3}
    Язык $L \subset \Sigma^*$ регулярен $\iff$ $\exists k > 0: \forall w \in \Sigma^*, len(w) \geq k$ 
    существуют $x,y,z \in \Sigma^*: w=xyz, y \neq \varepsilon$,
    и $\forall i \geq 0, \forall v \in \Sigma^*$ верно, что $wv \in L \iff xy^izv \in L$. 
\end{lemma}

Следует отметить, что в этой лемме нас фактически интересуют только $w$ -- префиксы слов из $L$, и $wv$ -- слова в $L$, 
так как в обратном случае всегда можно так разбить $w$ на $x,y,z$, что $xy^izv$ так же не будет словом из $L$.  

\begin{theorem}
    \label{condition_one_sided_loops}
    Если для $\forall T = T_1 T_2 T_3 T_4 T_5$, $T \in Core(G, 1, 1)$, $(T_2, T_4)$ -- простые парные циклы, 
    верно, что $\omega(T_2) = \varepsilon$ или $\omega(T_4) = \varepsilon$,
    то $G$ регулярен.
\end{theorem}

\begin{proof}
    Приведем граф $G$ в нормальную форму $G' = NormalForm(G)$. 
    Очевидно, свойство отстутсвия букв на одной части парных циклов сохранится.
    Пусть $k = max_{T \in Core(G',1,1)} len(\omega(T)) + 1$. 
    По \ref{canon_length_limit_theorem}, длина путей из ядра ограничена, значит $k$ существует и конечно.

    Воспользуемся \ref{pumping_lemma_4_3}.
    Рассмотрим префикс успешного пути $T'$, $\omega(T') = w, len(\omega(T')) \geq k$. 
    Так как $len(T') \geq len(\omega(T')) \geq k$, то в $T'$ обязательно есть цикл, 
    причем из выбора $k$, в этом цикле обязательно есть буква.
    Обозначим $T' = T_x T_y T_z$, где $\omega(T_x) = x, \omega(T_y) = y, \omega(T_z) = z$, $T_y$ -- цикл.
    
    Рассмотрим все такие $T''$, что $T = T' T''$ -- успешный путь в $G'$.
    Если $T_y$ -- нейтральный цикл, то условие $\omega(T' T'') \in L(G) \iff \omega(T_x T_y^i T_z T'') \in L(G)$ очевидно выполняется.
    Если $T_y$ -- правая часть парных циклов, то есть $T' = T_{x,1} T_{x,2} T_{x,3} T_y T_z$, $(T_{x,2}, T_y)$ -- парные,
    то существует префикс $T'^{(i)} = T_{x,1} T_{x,2}^i T_{x,3} T_y^i T_z$, при этом, из условия теоремы, $\omega(T_{x,2}) = \varepsilon$,
    и $\omega(T'^{(i)}) = \omega(T_x) \omega(T_y)^i \omega(T_z)$. Тогда $\omega(T' T'') \in L(G) \iff \omega(T'^{(i)} T'') \in L(G)$.
    В случае, если $T_y$ -- левая часть парного цикла, соответсвующая правая часть может быть в $T'$, а может и не быть. 
    Если $T' = T_x T_y T_{z,1} T_{z,2} T_{z,3}$, $(T_y,T_{z,2})$ -- парные, доказать можно, как и в случае $T_y$ -- правой части.
    Во втором случае, $T = T_x T_y T_z T_a T_b T_c$, $(T_y,T_b)$ -- парные,
    $\omega(T_b) = \varepsilon$. Сущетсвует путь $T^{(i)} = T_x T_y^i T_z T_a T_b^i T_c$, 
    причем $\omega(T^{(i)}) = \omega(T_x) \omega(T_y)^i \omega(T_z) \omega(T_a) \omega(T_c)$. Тогда $\omega(T) \in L(G) \iff \omega(T^{(i)}) \in L(G)$ $\Box$

\end{proof}

\begin{corollary}
    \label{codition_only_neutral}
    Пусть $Core(G, 1, 0) = Core(G, 1, 1)$.
    \begin{enumerate}[label=\arabic*)]
        \item Это эквивалентно $Core(G, 1, d) = Core(G, 1, 0), \forall d = 0,1,2 \dots$.
        \item Из этого следует, что $G$ -- регулярный.
    \end{enumerate}
\end{corollary}
Это значит, что в путях $Core(G, 1, 1)$ вообще нет парных циклов (а следовательно, и во всех остальных ядрах).


Все эти условия регулярности легко реализовать алгоритмически, пользуясь алгоритмом \ref{core_gen_algo}.

\clearpage