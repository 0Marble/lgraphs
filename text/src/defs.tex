\chapter{Основные понятия} \label{chapter1}

\section{Формальные языки}

\emph{Алфавит} $\Sigma = \{ a_1, a_2 \dots a_n \}$ -- непустое конечное множество \emph{символов}.

\emph{Цепочка} из символов $a_{i_1} a_{i_2} \dots a_{i_k}, a_{i_j} \in \Sigma$ -- некоторая последовательность символов.
Цепочку так же можно называть \emph{словом} или \emph{предложением}.

\emph{Звезда Клини} $\Sigma^*$ -- множество всех цепочек алфавита $\Sigma$.
\emph{Плюс Клини} $\Sigma^+$ -- множество всех непустых цепочек алфавита $\Sigma$.

\emph{Язык} L над алфавитом $\Sigma$ -- подмножество всех возможных цепочек из алфавита $\Sigma$.

\emph{Грамматикой} называется четверка $G = \langle T, N, P, S \rangle$, где
\begin{enumerate}[label=\arabic*)]
    \item $T$ -- множество символов-\emph{терминалов};
    \item $N$ -- множество символов-\emph{нетерминалов}, $N \cap T = \emptyset$;
    \item $S$ -- начальный нетерминал, $S \in N$;
    \item {
        $P$ -- множество правил вывода. $P \subseteq (T \cup N)^*N(T \cup N)^* \times (T \cup N)^*$.
        Правила принято записывать как $\alpha \rightarrow \beta$.
    }
\end{enumerate}

Цепочка $\beta \in (T \cup N)^*$ \emph{непосредственно выводима} из $\alpha \in (T \cup N)^+$ в 
грамматике $G = \langle T, N, P, S \rangle$,
если $\alpha = \xi_1 \gamma \xi_2$, $\beta = \xi_1 \delta \xi_2$, $\gamma \rightarrow \delta \in P$.
Обозначается $\alpha \rightarrow_G \beta$.

Цепочка $\beta \in (T \cup N)^*$ \emph{выводима} из $\alpha \in (T \cup N)^+$ в 
грамматике $G = \langle T, N, P, S \rangle$, если существует последовательность
$\psi_1 \psi_2 \dots \psi_n, \psi_i$, такая что $\alpha \rightarrow_G \psi_1 \rightarrow_G \dots \rightarrow_G \psi_n \rightarrow_G \beta$. Обозначается $\alpha \rightarrow_G^* \beta$.

Грамматике $G$ можно поставить в соответствие язык $L(G) = \{ \alpha \in T^* | S \rightarrow_G^* \alpha \}$. 

\emph{Регулярные языки} определяются как языки, с грамматикой с правилами вида $\alpha \rightarrow \beta$, $\alpha \in N$, $\beta \in T \cup TN \cup \varepsilon$.
Регулярные языки можно описывать с помощью \emph{конечных автоматов} -- наборов вида $\langle Q, \Sigma, \delta, q_0, F \rangle$, где
\begin{enumerate}[label=\arabic*)]
    \item $Q$ -- конечное множество состояний;
    \item $\Sigma$ -- алвавит;
    \item $\delta$ -- отоброжение из $Q \times \Sigma$ в $\mathcal{P} (Q)$ -- все подмножества $Q$, функция переходов;
    \item $q_0 \in Q$ -- начальная вершина;
    \item $F \subseteq Q$ -- множество конечных вершин.
\end{enumerate}

\emph{Контекстно-свободные языки} определяются как языки с грамматикой с правилами вида $\alpha \rightarrow \beta$, $\alpha \in N$. 

\section{Скобки}

% cSpell:ignore псевдоцикл биекция детирминированным псевдоциклом псевдоциклы

Пусть заданы множества $P_[, P_], P_[ \cap P_] = \emptyset$. 
$P_[$ -- множество \emph{открывающих скобок}, $P_]$ -- \emph{закрывающих скобок}.
Так же задана биекция $\phi(a) = b, a \in P_[, b \in P_]$.
$P = \left\{ (a, b) \mid a \in P_[, b \in P_], \phi(a) = b \right\} \subseteq \left( P_[ \times P_] \right)$ -- множество пар скобок.

Язык $L_p$, порождаемый грамматикой $S \rightarrow \varepsilon \mid a S b S, (a, b) \in P$ назовем \emph{языком Дика}. 
Он описывает правильные скобочные последовательности.

Введем отображение $\mu : (P_[ \cup P_])^* \to (P_[ \cup P_])^* $, показывающее \emph{баланс скобок}:
\begin{enumerate}[label=\arabic*)]
    \item $\mu(\varepsilon) = \varepsilon$;
    \item $\mu(P_1 a b P_2) = \mu(P_1 P_2)$, если $(a, b) \in P$;
    \item иначе, $\mu(X) = X$.
\end{enumerate}

\emph{Глубина} скобочной последовательности $depth(X)$ определяется следующими образом:
\begin{enumerate}[label=\arabic*)]
    \item $depth(\varepsilon) = 0$;
    \item $depth(a P_1 b) = depth(P_1) + 1$, если $(a, b) \in P$;
    \item иначе, $depth(P_1 P_2) = max(depth(P_1), depth(P_2))$.
\end{enumerate}

\section{L-граф}
L-графом назовем шестерку $\langle V,E,I,F,\Sigma,P\rangle$, где:
\begin{enumerate}[label=\arabic*)]
    \item $V$ -- множество \emph{вершин}, $Nodes(G) = V$;
    \item $I \subseteq V$ -- непустое множество \emph{начальных вершин}, $Init(G) = I$;
    \item $F \subseteq V$ -- непустое множество \emph{конечных вершин}, $Final(G) = F$;
    \item $\Sigma$ -- множество символов входного алфавита или пометок;
    \item $P$ -- множество пар скобок
    \item $E \subseteq V 
                    \times \left\{ \Sigma \cup \left\{ \varepsilon \right\}\right\} 
                    \times \left\{ P_[ \cup P_] \cup \left\{ \varepsilon \right\} \right\} 
                    \times V $
        -- множество \emph{дуг}, $Edges(G) = E$.
\end{enumerate}
Все множества конечные.

\begin{figure}
    \centering
    \includesvg[inkscapeformat=png]{images/graph1.dot.svg}
    \caption{L-граф}
    \label{lgraph1-expample-image}
    
\end{figure}

Для дуги $\pi = \langle s, a, b, t \rangle$ введем дополнительные обозначения:
\begin{enumerate}[label=\arabic*)]
    \item $beg(\pi) = s$ -- начальная вершина дуги;
    \item $end(\pi) = t$ -- конечная вершина дуги;
    \item $\omega(\pi) = a$ -- \emph{пометка} дуги;
    \item $\iota(\pi) = b$ -- \emph{скобочный след} дуги.
\end{enumerate}

\section{Пути в L-графе}
\emph{Путь} $T$ в L-графе $G=\langle V,E,I,F,\Sigma, P \rangle$ -- такая последовательность $p_0 \pi_1 p_1 \cdots p_n \pi_n p_n$, что 
$p_i \in V, \pi_i \in E, beg(\pi_i) = p_{i-1}, end(\pi_i) = p_{i+1}$.
Пустой путь состоит только из одной вершины $p_0$, а если путь не пустой, то его однозначно определяют его дуги.
Начало пути $beg(T) = p_0$, конец пути $end(T) = p_n$, $\left| T \right| = n$ -- длина пути, 
$Edges(T) = \left\{ \pi_i \right\}$, $Nodes(T) = \left\{ p_i \right\}$.

\emph{Пометка} пути $\omega(T)$ -- последовательность символов входного алфавита, которые расположены на дугах этого пути:
\begin{enumerate}[label=\arabic*)]
    \item Если $T$ -- пустой путь, то $\omega(T) = \varepsilon$;
    \item Если $T = \pi T_1$, $\pi$ -- дуга, $T_1$ -- маршрут, то $\omega(T) = \omega(\pi) \omega(T_1)$.
\end{enumerate}

\emph{Скобочный след} пути $\iota(T)$ определяется аналогично.


\emph{Успешный путь} -- такой путь $T$, что $beg(T) \in I, end(T) \in F, \mu(\iota(T)) = \varepsilon$.
В дальнейшем, под словом путь будем понимать успешные пути.

Любой фрагмент пути будем называть \emph{маршрутом}, для маршрутов определены все те же функции, что и для пути.
Если $T = T_1 T_2$ -- путь, то $T_1$ -- его маршрут-\emph{префикс}, $T_2$ -- \emph{суффикс}.

Язык $L(G)$, определяемый L-графом $G$ -- множество из всевозможных $\omega(T)$, где $T$ -- успешный путь.

\section{Память}
\emph{Памятью} будем называть пару $\langle q, \alpha \rangle$, $\alpha=\mu(\iota(T)), end(T)=q$, $T$ --  префикс пути в $G$. 
Введем отображение $Mem(T_1,T_2) = T_2', end(T_1) = beg(T_2)$, $\iota(T_1 T_2)$ является префиксом слова из $L_p$, $\omega(T_2) = \omega(T_2')$.
Оно определяется следующим образом:
\begin{enumerate}[label=\arabic*)]
    \item $Mem(T_1, p) = \langle p, \mu(\iota(T_1)) \rangle$, если $p \in Nodes(G)$;
    \item {
        $Mem(T_1, T_2) = \langle p_0', a_1, b_1, p_1' \rangle \cdots \langle p_{n-1}', a_n, b_n, p_n' \rangle$, если $T_2 =\pi_1 \cdots \pi_n$, 
        где $\pi_i = \langle p_{i-1}, a_i, b_i, p_i \rangle$ -- дуга $G$, $p_i' = end(Mem(T_1, \pi_1 \cdots \pi_i))$.
    }
\end{enumerate}

Для краткости будем писать $Mem(T) = Mem(p, T), p = beg(T)$, и называть такой маршрут маршрутом \emph{в форме памяти}. 

\begin{lemma}
    \label{mem_continue_lemma}
    Пусть $T=T_1 T_2$ -- путь в $G$, $end(T_1) = p$, $Mem(T_1, p) = p'$, тогда $\forall T_3 : beg(T_3) \in Init(G), Mem(T_3, p) = p' \implies T' = T_3 T_2$ -- успешный путь. 
\end{lemma}
\begin{proof}
    Пусть $\alpha = \mu(\iota(T_1)), \beta = \mu(\iota(T_2))$.
    $Mem(T_3, p) = p' \implies \mu(\iota(T_3)) = \alpha$.
    Это значит, что $\forall T_4 : \mu(\alpha \iota(T_4)) = \varepsilon$, $end(T_4)$ -- конечная вершина  $\implies T_3 T_4$ -- успешный путь.
    $T$ -- успешный $\implies \mu(\alpha \beta) = \varepsilon$, $end(T_2)$ -- конечная вершина $G$. 
    То есть, при $T_4 = T_2, T' = T_3 T_2$ -- успешный $\Box$ 
\end{proof}

\section{Детерминированность}
Определим функцию \begin{eqnarray*}
    direct(\pi) & = & \left\{ a \in \Sigma \mid \exists T_1 \pi T_2 - \text{успешный путь}, \omega(\pi T_2) = a \alpha, \alpha \in \Sigma^* \right\} \\
    & \cup & \left\{ \varepsilon \mid \exists T_1 \pi T_2 - \text{успешный путь}, \omega(\pi T_2) = \varepsilon \right\}
\end{eqnarray*}
L-граф $G$ называется \emph{детeрминированным}, если
$\forall \pi_1, \pi_2 \in Edges(G), \pi_1 \neq \pi_2, beg(\pi_1) = beg(\pi_2)$, 
из условия $direct(\pi_1) \cap direct(\pi_2) = \emptyset$ следует, что 
$\iota(\pi_1), \iota(\pi_2) \in P_] \cap \iota(\pi_1) \neq \iota(\pi_2)$. 

\begin{figure}[h]
    \begin{minipage}[h]{0.49\linewidth}
      \center{\includesvg[scale=0.7, inkscapeformat=png]{images/example_determined.dot.svg} \caption{Детерминированный L-граф}}
    \end{minipage}
    \hfill
    \begin{minipage}[h]{0.49\linewidth}
      \center{\includesvg[scale=0.7, inkscapeformat=png]{images/example_nondetermined.dot.svg} \caption{Недетерминированный L-граф}}
    \end{minipage}
    \label{det_vs_nondet}  
  \end{figure}

Неформально это можно понимать как то, что, находясь в любой вершине, 
по верхней скобке в стеке и следующей букве всегда можно однозначно определить, по какой дуге идти дальше.

\section{Циклы}

\emph{Циклом} в L-графе будем называть такой маршрут $T$, где $beg(T) = end(T)$. 
В L-графе можно определить 3 типа циклов: \emph{нейтральные циклы, парные циклы и псевдоциклы}.
\emph{Нейтральный цикл} -- это такой цикл $T$, у которого $\mu(\iota(T)) = \varepsilon$.

\begin{figure}
    \centering
    \subfloat[\centering Граф с нейтральным циклом]{
        \includesvg[inkscapeformat=png]{images/example_neutral.dot.svg}
    }
    \qquad
    \subfloat[\centering Граф с парными циклами]{
        \includesvg[inkscapeformat=png]{images/example_paired.dot.svg}
    }
    \qquad
    \subfloat[\centering Граф с псевдоциклом]{
        \includesvg[inkscapeformat=png]{images/example_pseudo.dot.svg}
    }
    \caption{3 вида циклов}
    \label{loop-kinds-example}
\end{figure}


Для определения парных циклов и псевдоциклов, нам понадобится дополнительное понятие гнезда.

Пусть в пути $T$ есть маршрут $T_1 T_2 T_3$. 
Тройку $\langle T_1, T_2, T_3 \rangle$, $T_1, T_3$ -- не пустые и не нейтральные маршруты, назовем \emph{гнездом}, 
если $\mu(\iota(T_2)) = \mu(\iota(T_1 T_2 T_3)) = \varepsilon$.
Если в гнезде $\langle T_1, T_2, T_3 \rangle$ $T_1$ и $T_3$ -- циклы, то будем их называть \emph{парными}.
Если только $T_1$ (или только $T_3$) -- цикл, то будем считать его \emph{псевдоциклом}.

Парные циклы пути $T = T_1 T_2 T_3 T_4 T_5$ $\langle T_2, T_3, T_4 \rangle$ будем называть \emph{простыми}, если
путь $T$ нельзя представить в виде $T_1 T_{21} T_{22} T_3 T_{42} T_{41} T_5$, где $T_2 = T_{21} T_{22}, T_4 = T_{42} T_{41}$,
$\langle T_{22}, T_3, T_{42} \rangle$ -- парные. Фактически это значит, что $\langle T_2, T_3, T_4\rangle$ является
\emph{минимальным}, и из него нельзя выделить более простые парные циклы.

\section{Ядро L-графа}

\emph{$(w,d)$-каноном} будем называть успешный путь $T$ со следующими свойствами:
\begin{enumerate}[label=\arabic*)]
    \item Для любого маршрута $T_1 T_2 \cdots T_n$, где все $T_i$ -- нейтральные циклы, $n \leq w$;
    \item Для любого маршрута $T_{l,n} T_{l,n-1} \cdots T_{l,1} T_{m,1} T_{r,1} T_{r,2} \cdots T_{r,n}$,\\
       $T_{m,i+1} = T_{l,i} T_{m,i} T_{r,i}$, $\langle T_{l,i}, T_{m,i}, T_{r,i} \rangle$ -- гнезда, $T_{l,i}, T_{r,i}$ -- циклы,
       $n \leq d$.
\end{enumerate}

\emph{Ядром} L-графа $Core(G, w, d)$ будем называть множество всех $(w,d)$-канонов. 
Иначе говоря, $Core(G, w, d)$ -- это множество всех успешных путей, 
где количество подряд идущих нейтральных циклов не превышает $w$, 
количество вложенных подряд идущих парных циклов не превышает $d$.  

Одно важное свойство ядра: $w_1 \leq w_2, d_1 \leq d_2 \implies Core(w_1, d_1) \subseteq Core(w_2, d_2)$

\begin{figure}[h]
    \centering
    \includesvg[scale=.65, inkscapeformat=png]{images/graph1_path0.dot.svg} (a)\\
    \includesvg[scale=.65, inkscapeformat=png]{images/graph1_path3.dot.svg} (b)\\
    \includesvg[scale=.65, inkscapeformat=png]{images/graph1_path1.dot.svg} (c)\\
    \includesvg[scale=.65, inkscapeformat=png]{images/graph1_path4.dot.svg} (d)\\
    
    \caption{Пути из различных ядер графа \ref{lgraph1-expample-image}. (a) $\in Core(0,0)$, (b) $\in Core(1,0)$, (c),(d) $\in Core(1,1)$}
    \label{lgraph1-core-example}
\end{figure}

\section{Нормальная форма L-графа}

L-граф $G$ будем считать L-графом \emph{в нормальой форме}, если в его путях нет \emph{псевдоциклов}.
Псевдоциклы доставляют некоторые трудности при рассуждениях об L-графах, так как, к примеру,
некоторые циклические части графа на самом деле не приводят к зацикливанию, или, что еще хуже, могут
быть частью парнымых циклов в неоторых путях, а в некоторых -- псевдоциклов. 

\clearpage