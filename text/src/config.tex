
\usepackage{cmap}
% \usepackage[T1, T2A]{fontenc}

\usepackage{fontspec}

\usepackage[utf8]{inputenc}
\usepackage[english, russian]{babel}

\usepackage[UTF8]{ctex}%中文
\usepackage{tabularray}%+
\newcommand{\song}{\songti}    % 宋体
\newcommand{\fs}{\fangsong}        % 仿宋体
\newcommand{\kai}{\kaishu}      % 楷体
\newcommand{\hei}{\heiti}      % 黑体
\newcommand{\li}{\lishu}        % 隶书
\newcommand{\yihao}{\fontsize{26pt}{26pt}\selectfont}       % 一号, 单倍行距
\newcommand{\xiaoyi}{\fontsize{24pt}{24pt}\selectfont}      % 小一, 单倍行距
\newcommand{\erhao}{\fontsize{22pt}{1.25\baselineskip}\selectfont}       % 二号, 1.25倍行距
\newcommand{\xiaoer}{\fontsize{18pt}{18pt}\selectfont}      % 小二, 单倍行距
\newcommand{\sanhao}{\fontsize{16pt}{16pt}\selectfont}      % 三号, 单倍行距
\newcommand{\xiaosan}{\fontsize{15pt}{15pt}\selectfont}     % 小三, 单倍行距
\newcommand{\sihao}{\fontsize{14pt}{14pt}\selectfont}       % 四号, 单倍行距
\newcommand{\xiaosi}{\fontsize{12pt}{12pt}\selectfont}      % 小四, 单倍行距
\newcommand{\wuhao}{\fontsize{10.5pt}{10.5pt}\selectfont}   % 五号, 单倍行距
\newcommand{\xiaowu}{\fontsize{9pt}{9pt}\selectfont}        % 小五, 单倍行距
\makeatletter
\newcommand\dlmu[2][4cm]{\hskip1pt\underline{\hb@xt@ #1{\hss#2\hss}}\hskip3pt}
\makeatother



\usepackage{amsmath}
\usepackage{graphicx}
\usepackage{algorithmicx}
\usepackage{algpseudocode}
\usepackage{subcaption}
\usepackage{indentfirst}
\usepackage{cite}
\usepackage[linktocpage=true,plainpages=false,pdfpagelabels=false]{hyperref}
\usepackage[usenames]{color}
\usepackage{color}
\usepackage{colortbl}
\usepackage{tocloft}
\usepackage{lscape}		% Для включения альбомных страниц
\usepackage{geometry}	% Для последующего задания полей
\usepackage{svg}
\usepackage{minted}
\usepackage{array}
\usepackage{multirow}
\usepackage{enumitem}
\usepackage{amssymb}

\newtheorem{theorem}{Теорема}[section]
\newtheorem{lemma}[theorem]{Лемма}
\newtheorem{proposition}[theorem]{Предположение}

\newenvironment{proof}[1][Доказательство]{\begin{trivlist}
\item[\hskip \labelsep {\bfseries #1}]}{\end{trivlist}}
\newenvironment{definition}[1][Определение]{\begin{trivlist}
\item[\hskip \labelsep {\bfseries #1}]}{\end{trivlist}}
\newenvironment{example}[1][Пример]{\begin{trivlist}
\item[\hskip \labelsep {\bfseries #1}]}{\end{trivlist}}
\newenvironment{remark}[1][Замечание]{\begin{trivlist}
\item[\hskip \labelsep {\bfseries #1}]}{\end{trivlist}}

\textheight=23cm
\textwidth=16cm
\oddsidemargin=5mm % левое поле для нечётных страниц
\evensidemargin=-5mm % севое поле для чётных страниц
%\marginparwidth=36pt
\topmargin=-1cm % расстояние от верхней границы листа до заголовка
%\flushbottom % выстота тела всех страниц одинаковая 
\raggedbottom % позволяет несколько различаться высоте тел различных страниц
\tolerance 3000 % how much badness is allowable without error. 

\linespread{1.3} %1.5 spacing 

\parindent 1.27cm % Абзацный отступ

\sloppy             % текст редко залезает на правое поле
\clubpenalty = 10000  % Запрещаем разрыв страницы после первой строки абзаца
\widowpenalty = 10000 % Запрещаем разрыв страницы после предпоследней строки абзаца
% \global\hyphenpenalty = 1000 % Частота переносов

\setmainfont{Times New Roman}

%%% Библиография %%%
\makeatletter
\bibliographystyle{src/utf8gost705u}	% Оформляем библиографию в соответствии с ГОСТ 7.0.5
\renewcommand{\@biblabel}[1]{#1.}	% Заменяем библиографию с квадратных скобок на точку:
\makeatother


%%% Цвета гиперссылок %%%
\definecolor{linkcolor}{rgb}{0.9,0,0}
\definecolor{citecolor}{rgb}{0,0.6,0}
\definecolor{urlcolor}{rgb}{0,0,1}
\hypersetup{
    colorlinks, linkcolor={linkcolor},
    citecolor={citecolor}, urlcolor={urlcolor}
}

%%% Оглавление %%%
\renewcommand{\cftchapdotsep}{\cftdotsep}
