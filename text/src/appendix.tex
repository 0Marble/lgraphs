%cSpell: ignore Modifyable, usize, loopify, concat

\appendix
\chapter{Описание интерфейсов библиотеки} \label{AppendixA}

\begin{longtable}[c]{|m{4em}|m{9em}|m{10em}|m{10em}|}
    \caption{Список классов.\label{all_class_table}}\\

    \hline
        \textbf{Ссылка на таблицу методов} & \textbf{Название класса} & \textbf{Наследование} & \textbf{Описание}\\
    \hline
        -- & Letter & Self: Debug + Clone + PartialEq + Eq + Hash & Интерфейс для символов входного алфавита \\
    \hline
        -- & Node & Self: Debug + Clone + PartialEq + Eq + Hash  & Интерфейс для названий вершин \\
    \hline
        \ref{class_Bracket_api} & Bracket & Self: Debug + Clone + PartialEq + Eq + Hash  & Класс для описания скобок \\
    \hline
        \ref{class_BracketStack_api} & BracketStack & Self: Debug + Clone + PartialEq + Eq + Hash + Default & Класс для описания стека скобок \\
    \hline
        \ref{class_EdgeNL_api} \newline \ref{class_EdgeNLGraphLetterL_api} & Edge<N, L> & Self: Debug + Clone + PartialEq + Eq + Hash \newline N: Node \newline  L: Letter & Класс для описания граней графа \\
    \hline
        \ref{class_PathNL_api} \newline \ref{class_PathNLGraphLetterL_api} & Path<N, L> & Self: Debug + Clone + PartialEq + Eq + Hash + FromStr + Display \newline N: Node \newline L: Letter & Класс для описания путей и маршрутов в графах \\
    \hline
        \ref{class_GraphNL_api} & Graph<N, L> & N: Node \newline L: Letter & Интерфейс для описания графов \\
    \hline
        \ref{class_ModifyableGraphNL_api} & ModifyableGraph<N, L> & Self: Graph<N,L> \newline N: Node \newline L: Letter & Интерфейс для описания графов, которые можно изменять \\
    \hline
        \ref{class_DefaultGraphNL_api} & DefaultGraph<N, L> & Self: Clone + Debug + Display + FromStr + Graph<N, L> + ModifyableGraph<N, L> + PartialEq \newline N: Node \newline L: Letter & Имплементация Graph по умолчанию \\
    \hline
        \ref{class_LGraphLetterL_api} & LGraphLetter<L> & Self: Clone + Debug + Default + Display + Eq + FromStr + Hash + Letter + PartialEq \newline L: Letter & Пара (буква, скобка) \\
    \hline
        \ref{class_MemoryN_api} & Memory<N> & Self: Clone + Debug + Display + Eq + FromStr + Hash + Node + PartialEq \newline N: Node & Память L-графа, пара (вершина, стек скобок) \\
    \hline
        \ref{class_LGraphGNL_api} & LGraph<G, N, L> & Self: Clone + Debug + Display + Eq + FromStr + Graph<N, LGraphLetter<L>{}> + PartialEq \newline G: ModifyableGraph<N, LGraphLetter<L>{}> \newline N: Node \newline L: Letter & Класс для описания L-графов\\
    \hline
\end{longtable}

\begin{table}
    \caption{Методы класса Bracket}
    \label{class_Bracket_api}
    \begin{tabular}{|m{20em}|m{20em}|}
        \hline
        \textbf{Метод} & \textbf{Описание}\\
        \hline
            fn new(index: usize, open: bool) -> Self & Конструктор \\
        \hline
            fn index(\&self) -> usize & Индекс скобки \\
        \hline
            fn is\_open(\&self) -> bool & Открывающая или закрывающая скобка \\
        \hline
    \end{tabular}
\end{table}
    

\begin{table}
    \caption{Методы класса BracketStack}
    \label{class_BracketStack_api}
    \begin{tabular}{|m{20em}|m{20em}|}
        \hline
        \textbf{Метод} & \textbf{Описание}\\
        \hline
            fn new() -> Self & Конструктор \\
        \hline
            fn accept(\&mut self, bracket: Bracket) & Добавит скобку на верх стека \\
        \hline
            fn can\_accept(\&self, bracket: \&Bracket) -> bool & Проверить, можно ли добавить скобку в стек \\
        \hline
            fn clear(\&mut self) & Очистить стек \\
        \hline
            fn state(\&self) -> \&[usize] & Список индексов скобок в стеке \\
        \hline
            fn is\_empty(\&self) -> bool & Пустой ли стек \\
        \hline
            fn len(\&self) -> usize & Сколько скобок в стеке \\
        \hline
            fn from\_brackets(brackets: Vec<Bracket>) -> Option<Self> & Попробовать создать стек из списка скобок \\
        \hline
    \end{tabular}
\end{table}

\begin{table}
    \caption{Методы класса Edge<N, L>}
    \label{class_EdgeNL_api}
    \begin{tabular}{|m{20em}|m{20em}|}
        \hline
        \textbf{Метод} & \textbf{Описание}\\
        \hline
            fn new(source: N, target: N, letter: L) -> Self & Конструктор \\
        \hline
            fn beg(\&self) -> \&N & Начальная вершина дуги \\
        \hline
            fn end(\&self) -> \&N & Конечная вершина дуги \\
        \hline
            fn letter(\&self) -> \&L & Буква на дуге \\
        \hline
    \end{tabular}
\end{table}

\begin{table}
    \caption{Методы класса Edge<N, LGraphLetter<L>{}>}
    \label{class_EdgeNLGraphLetterL_api}
    \begin{tabular}{|m{20em}|m{20em}|}
        \hline
        \textbf{Метод} & \textbf{Описание}\\
        \hline
        fn item(\&self) -> Option<\&L> & Буква на дуге \\
        \hline
            fn bracket(\&self) -> Option<\&Bracket> & Скобка на дуге \\
        \hline
    \end{tabular}
\end{table}

\begin{table}
    \caption{Методы класса Path<N, L>}
    \label{class_PathNL_api}
    \begin{tabular}{|m{20em}|m{20em}|}
        \hline
        \textbf{Метод} & \textbf{Описание}\\
        \hline
            fn new(node: N) -> Self & Конструктор. Изначально, пустой путь состоит из одной вершины. \\
        \hline
            fn add\_edge(\&mut self, edge: Edge<N, L>) & Добавить дугу \\
        \hline
            fn beg(\&self) -> \&N & Начало пути \\
        \hline
            fn end(\&self) -> \&N & Конец пути \\
        \hline
            fn len\_in\_edges(\&self) -> usize & Длина пути в дугах \\
        \hline
            fn len\_in\_nodes(\&self) -> usize & Длина пути в вершинах \\
        \hline
            fn is\_empty(\&self) -> bool & Пустой ли путь \\
        \hline
            fn nth\_node(\&self, n: usize) -> Option<\&N> & Вершина с индексом $n$ \\
        \hline
            fn nth\_edge(\&self, n: usize) -> Option<\&Edge<N, L>{}> & Дуга с индексом $n$ \\
        \hline
            fn nodes(\&self) -> impl Iterator<Item = \&N> + '\_ & Итератор по всем вершинам \\
        \hline
            fn edges(\&self) -> impl Iterator<Item = \&Edge<N, L>{}> + '\_ & Итератор по всем дугам \\
        \hline
            fn subpath(\&self, beg\_node\_index: usize, end\_node\_index: usize) -> Option<Self> & Маршрут пути от вершины с индексом beg\_node\_index до вершины с индексом end\_node\_index включительно \\
        \hline
            fn loopify\_on\_first(\&self) -> Option<Self> & Заменяет последнюю вершину на первую, создавая цикл \\
        \hline
            fn loopify\_on\_last(\&self) -> Option<Self> & Заменяет первую вершину на последнюю, создавая цикл \\
        \hline
            fn concat(\&self, next: \&Self) -> Self & Конкатенация путей \\
        \hline
            fn to\_dot(\&self) -> String \newline where N: ToString, L:ToString & Записывает путь в формате DOT \\
        \hline
    \end{tabular}
\end{table}

\begin{table}
    \caption{Методы класса Path<N, LGraphLetter<L>{}>}
    \label{class_PathNLGraphLetterL_api}
    \begin{tabular}{|m{20em}|m{20em}|}
        \hline
        \textbf{Метод} & \textbf{Описание}\\
        \hline
            fn mem(\&self) -> Path<Memory<N>, LGraphLetter<L>{}> & $Mem(T)$ \\
        \hline
            fn depth(\&self) -> usize & $depth(T)$ \\
        \hline
            fn get\_w(\&self) -> usize & Параметр $w$ для ядра \\
        \hline
            fn get\_d(\&self) -> usize & Параметр $d$ для ядра \\
        \hline
            fn iota(\&self) -> Vec<Bracket> & $\iota(T)$, скобочный след \\
        \hline
            fn is\_balanced(\&self) -> bool & Сбалансирован ли путь, $\mu(\iota(T)) = \varepsilon$ \\
        \hline
            fn is\_subpath\_balanced(\&self, beg\_node\_index: usize, end\_node\_index: usize) -> bool & Сбалансирован ли маршрут от beg\_node\_index до end\_node\_index включительно \\
        \hline
            fn paired\_loops(\&self) -> impl Iterator<Item = ((usize, usize), (usize, usize))> + '\_ & Итератор по всем парным циклам $(T_l, T_r)$, в формате $((l_1,l_2), (r_1, r_2))$, где $T_l = T.subpath(l_1, l_2)$, $T_r = T.subpath(r_1, r_2)$ \\
        \hline
            fn is\_nest(\&self, left: (usize, usize), right: (usize, usize)) -> bool & Образуют ли маршруты с индексами left и right гнездо\\
        \hline
            fn is\_simple\_paired\_loops(\&self, left: (usize, usize), right: (usize, usize)) -> bool & Является ли данный парный цикл простым \\
        \hline
            fn simple\_paired\_loops(\&self) -> impl Iterator<Item = ((usize, usize), (usize, usize))> + '\_ & Итератор по всем простым парным циклам \\
        \hline
    \end{tabular}
\end{table}

\begin{table}
    \caption{Методы интерфейса Graph<N, L>}
    \label{class_GraphNL_api}
    \begin{tabular}{|m{20em}|m{20em}|}
        \hline
        \textbf{Метод} & \textbf{Описание}\\
        \hline
            fn nodes(\&self) -> Box<dyn Iterator<Item = \&N> + '\_> & Итератор по всем вершинам графа \\
        \hline
            fn edges(\&self) -> Box<dyn Iterator<Item = \&Edge<N, L>{}> + '\_> & Итератор по всем дугам графа \\
        \hline
            fn start\_node(\&self) -> \&N & Начальная вершина \\
        \hline
            fn end\_nodes(\&self) -> Box<dyn Iterator<Item = \&N> + '\_> & Итератор по конечным вершинам \\
        \hline
            fn node\_count(\&self) -> usize & Количество вершин \\
        \hline
            fn is\_start\_node(\&self, node: \&N) -> bool & Является ли данная вершина начальной \\
        \hline
            fn is\_end\_node(\&self, node: \&N) -> bool & Является ли данная вершина конечной \\
        \hline
            fn edges\_from<'a, 'b>(\&'a self, node: \&'b N) -> Box<dyn Iterator<Item = \&Edge<N, L>{}> + 'a> \newline 
            where L: 'a, 'b: 'a, & Итератор по всем дугам, начинающимся с вершины node \\
        \hline
            fn edges\_to<'a, 'b>(\&'a self, node: \&'b N) -> Box<dyn Iterator<Item = \&Edge<N, L>{}> + 'a> \newline 
            where L: 'a, 'b: 'a, & Итератор по всем дугам, кончающемся вершиной node\\
        \hline
            fn has\_node(\&self, node: \&N) -> bool & Есть ли данная вершина в графе \\
        \hline
            fn has\_edge(\&self, edge: \&Edge<N, L>) -> bool & Есть ли данная дуга в графе \\
        \hline
            fn to\_dot(\&self) -> String \newline where N: ToString, L: ToString, & Вывести граф в формате DOT \\
        \hline
    \end{tabular}
\end{table}

\begin{table}
    \caption{Методы интерфейса ModifyableGraph<N, L>}
    \label{class_ModifyableGraphNL_api}
    \begin{tabular}{|m{20em}|m{20em}|}
        \hline
        \textbf{Метод} & \textbf{Описание}\\
        \hline
            fn new\_empty(start\_node: N, end\_nodes: impl IntoIterator<Item = N>) -> Self & Конструктор графа без дуг \\
        \hline
            fn add\_edge(\&mut self, edge: Edge<N, L>) & Добавить дугу \\
        \hline
            fn from\_paths(paths: impl IntoIterator<Item = Path<N, L>{}>) -> Option<Self> \newline where Self: Sized, & Создать граф из данных путей \\
        \hline
    \end{tabular}
\end{table}

\begin{table}
    \caption{Методы класса DefaultGraph<N, L>}
    \label{class_DefaultGraphNL_api}
    \begin{tabular}{|m{20em}|m{20em}|}
        \hline
        \textbf{Метод} & \textbf{Описание}\\
        \hline
            fn new(start\_node: N, end\_nodes: impl IntoIterator<Item = N>) -> Self & Конструктор. Для добавления дуг, пользоваться методами из ModifyableGraph \\
        \hline
    \end{tabular}
\end{table}

\begin{table}
    \caption{Методы класса LGraphLetter<L>}
    \label{class_LGraphLetterL_api}
    \begin{tabular}{|m{20em}|m{20em}|}
        \hline
        \textbf{Метод} & \textbf{Описание}\\
        \hline
            fn new(letter: Option<L>, bracket: Option<Bracket>) -> Self & Конструктор \\
        \hline
            fn letter(\&self) -> Option<\&L> & Ссылка на букву \\
        \hline
            fn bracket(\&self) -> Option<\&Bracket> & Ссылка на скобку \\
        \hline
    \end{tabular}
\end{table}

\begin{table}
    \caption{Методы класса Memory<N>}
    \label{class_MemoryN_api}
    \begin{tabular}{|m{20em}|m{20em}|}
        \hline
        \textbf{Метод} & \textbf{Описание}\\
        \hline
            fn new(node: N, brackets: BracketStack) -> Self & Конструктор \\
        \hline
            fn node(\&self) -> \&N & Ссылка на вершину \\
        \hline
            fn brackets(\&self) -> \&BracketStack & Ссылка на стек скобок \\
        \hline
    \end{tabular}
\end{table}

\begin{table}
    \caption{Методы класса LGraph<G, N, L>}
    \label{class_LGraphGNL_api}
    \begin{tabular}{|m{20em}|m{20em}|}
        \hline
        \textbf{Метод} & \textbf{Описание}\\
        \hline
            fn new(graph: G) -> Self & Конструктор \\
        \hline
            fn graph(\&self) -> \&G & Ссылка на лежащий в основе граф \\
        \hline
            fn core<'a>(\&'a self, w: usize, d: usize) -> impl Iterator<Item = Path<N, LGraphLetter<L>{}>{}> + 'a \newline
            where Self: Sized, L: 'a, N: 'a, & Итератор по всем путям из $Core(G, w, d)$ \\
        \hline
            fn normal\_form<G0>(\&self) -> LGraph<G0, Memory<N>, L> \newline 
            where G0: ModifyableGraph<Memory<N>, LGraphLetter<L>{}>, & Перевод в нормальную форму \\
        \hline
            fn no\_brackets(\&self) -> bool & Проверка условия регулярности \ref{condition_no_brackets} \\
        \hline
            fn no\_letters(\&self) -> bool & Проверка условия регулярности \ref{condition_no_letters} \\
        \hline
            fn core11\_no\_letters\_on\_loops(\&self) -> bool & Проверка условия регулярности \ref{condition_one_sided_loops} \\
        \hline
    \end{tabular}
\end{table}
