\chapter*{Аннотация}

Приведены и доказаны некоторые достаточные условия регулярности КС-языков, описанных с помощью L-графов.
Разработан алгоритм приведения L-графа к нормальной форме, которая может быть удобна для дальнейших
теоретических исследований.
~ \\
~ \\
针对测试上下文无关语言正则性的问题,由于现有的条件很难在算法上进行测试,提出L-图这个形式语言中较新的概念,它易于理解且易于在程序中实现。
论文使用L-图来描述上下文无关语言,讨论如何确定上下文无关语言是否正则的问题。
论文基于L-图中核的概念,给出并证明了L-图正则性的几个充分条件,同时给出了检验这些条件的算法。
此外,论文中开发了一种生成L-图范式的算法,可能对未来正则性的相关研究提供帮助。
论文中出现的所有算法均被收集到一个用Rust编写的库中。
~ \\
~ \\
关键词:上下文无关语言; 正则语言; L-图; 范式; 核

\pagebreak