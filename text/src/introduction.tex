\chapter{Введение} \label{chapter0}

% cSpell: ignore Стирнзом

Теория формальных языков появилась в середине XX века. 
Формальные языки принято разделять на несколько классов, среди которых самые часто используемые на практике -- 
регулярные и контекстно-свободные (КС) языки. 

Регулярные языки часто представляются в виде регулярных выражений или
конечных автоматов, и применяются для большого количества задач обработки текста, 
таких как проверка правильности написания номеров телефона, адресов почты, дат и т.д.
Благодаря простоте и компактности регулярных выражений, их можно встретить почти в любом
приложении, а конечные автоматы являются одним из самых распространенных паттернов проектирования.

Контекстно-свободные языки обычно описываются с помощью КС грамматик,
и широко применяются для обработки и анализа текста программ, файлов конфигурации, математических
формул, и всего, в чем присутствуют скобки.

Класс контекстно-свободных языков шире класса регулярных, однако это делает их более сложными,
и с теоретической, и с практической точки зрения. К примеру, при работе конечного автомата,
в памяти требуется хранить только сам автомат, и вершину, в которой находится автомат на данный 
момент. В КС языках же еще требуется следить за состоянием стека скобок,
что значит, занимаемая в процессе алгоритма память может быть не ограниченна. 

Возникает вопрос, как понять, является ли данный КС язык на самом деле регулярным, что позволило бы
значительно упростить работу с ним. Этот вопрос не новый, и существует большое количество
достаточных условий регулярности КС языков, а так же критерий регулярности для детерминированных КС
языков, полученный Стирнзом в 60-ых в \cite{stearns_reg}.

Задачей данной работы является выведение и реализация способов проверки регулярности для альтернативного 
метода описания КС языков -- L-графов \cite{vylitok_rostovski_o_podklassah,vylitok_sutirin_harakterizacia}, являющихся эволюцией D-графов, описанных в \cite{stan1}.

\clearpage
