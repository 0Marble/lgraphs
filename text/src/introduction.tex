\chapter{Введение} \label{chapter0}

% cSpell: ignore Стирнзом

Теория формальных языков появилась в середине XX века. 
Формальные языки принято разделять на несколько классов, среди которых самые часто используемые на практике -- 
регулярные и контекстно-свободные (КС) языки. 

Регулярные языки часто представляются в виде регулярных выражений или
конечных автоматов, и применяются для большого количества задач обработки текста, 
таких как проверка правильности написания номеров телефона, адресов почты, дат и т.д.
Благодаря простоте и компактности регулярных выражений, их можно встретить почти в любом
приложении, а конечные автоматы являются одним из самых распространенных паттернов проектирования.

Контекстно-свободные языки обычно описываются с помощью КС-грамматик,
и широко применяются для обработки и анализа текста программ, файлов конфигурации, математических
формул, и всего, в чем присутствуют скобки.

Класс контекстно-свободных языков шире класса регулярных, однако это делает их более сложными,
и с теоретической, и с практической точки зрения. К примеру, при работе конечного автомата,
в памяти требуется хранить только сам автомат, и вершину, в которой находится автомат на данный 
момент. В КС-языках же еще требуется следить за состоянием стека скобок,
что значит, занимаемая в процессе алгоритма память может быть не ограничена. 

Возникает вопрос, как понять, является ли данный КС язык на самом деле регулярным, что позволило бы
значительно упростить работу с ним. Этот вопрос не новый, и существует большое количество
достаточных условий регулярности КС-языков, а так же критерий регулярности для детерминированных КС-
языков, полученный Стирнзом в 60-ых в \cite{STEARNS1967323}. Этот критерий показывает, что число состояний эквивалентного
магазинному автомату конечного автомата не превышает $t^{q^{q^{q}}}$,
где $t$ -- число символов магазинного автомата, $q$ -- число состояний.
Очевидно, такую оценку фактически невозможно применить на практике.
В работе Валианта \cite{Valiant1975RegularityAR} удалось получить оценку лучше -- с 2 степенями, 
но это число все равно очень большое. 

Еще одним важным инструментом в работе с формальными языками являются так называемые нормальные формы.
Они обычно представляют из себя специальные описания языков с определенными полезными свойствами.
К таким нормальным формам можно отнести нормальную форму Хомского, нормальную форму Грейбах,
которые упрощают доказательства теорем. Подробнее о номальных формах можно прочитать в \cite{handbook_of_formal}.
Разработка новых нормальных форм может помочь найти более простые критерии регулярности, хотя бы для
отдельных классов КС-языков.

Поиск новых условий регулярности КС-языков остается востребованным, нахождение
хороших методов доказательства имеет и практическую, и теоретическую ценность. 

\section{Постановка задачи}

Задачей данной работы является выведение и реализация способов проверки регулярности для альтернативного 
метода описания КС-языков -- L-графов \cite{vylitok_rostovski_o_podklassah,vylitok_sutirin_harakterizacia}, 
являющихся эволюцией D-графов, описанных в \cite{stan1}. Так же, будет приведен алгоритм построения нормальной формы
L-графа, который может помочь в дальнейших поисках и исследованиях условий регулрности.

Для достижения этих целей, понадобится решить следующие задачи.

\begin{enumerate}
    \item Изучить основную теорию, связанную с L-графами.
    \item Найти достаточные условия регулярности L-графов, реализуемые алгоритмически.
    \item Разработать алгоритм генерации ядра L-графа, и обосновать его корректность.
    \item Ввести понятие нормальной формы L-графа, и обосновать ее теоретический потенциал.
    \item Разработать алгоритм приведения L-графа к нормальной форме.
    \item Разработать библиотеку алгоритмов генрации ядра, приведения к нормальной форме и проверки условий регулярности для L-графов. 
\end{enumerate}

\clearpage
